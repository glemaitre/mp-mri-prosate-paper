%%%%%%%%%%%%%%%%%%%%%%%%%%%%%%%%%%%%%%%%%%%%%%%%%%%%%%%
% A template for Wiley article submissions.
% Developed by Overleaf. 
%
% Please note that whilst this template provides a 
% preview of the typeset manuscript for submission, it 
% will not necessarily be the final publication layout.
%
% Usage notes:
% The "blind" option will make anonymous all author, affiliation, correspondence and funding information.
% Use "num-refs" option for numerical citation and references style.
% Use "alpha-refs" option for author-year citation and references style.

\documentclass[num-refs]{wiley-article}
% \documentclass[blind,alpha-refs]{wiley-article}

% Add additional packages here if required
\input{file_system/package.tex}

% Update article type if known
\papertype{Original Article}
% Include section in journal if known, otherwise delete
\paperfield{Journal Section}

\title{Computer-aided detection and diagnosis using Multi-modal MRI for prostate cancer detection}

% List abbreviations here, if any. Please note that it is preferred that abbreviations be defined at the first instance they appear in the text, rather than creating an abbreviations list.
% \abbrevs{ABC, a black cat; DEF, doesn't ever fret; GHI, goes home immediately.}

% Include full author names and degrees, when required by the journal.
% Use the \authfn to add symbols for additional footnotes and present addresses, if any. Usually start with 1 for notes about author contributions; then continuing with 2 etc if any author has a different present address.
% \author[1\authfn{1}]{Author One PhD}
% \author[2\authfn{1}]{Author A.~Two MD}
% \author[2\authfn{2}]{Author Three PhD}
% \author[2]{Author B.~Four}
\author[1\authfn{1}]{G.~Lemaitre PhD}
\author[2\authfn{1}]{F.~Meriaudeau PhD}
\author[2\authfn{1}]{A.~Meyer-Bease PhD}
\author[2\authfn{1}]{R.~Marti PhD}
\author[2\authfn{1}]{K.~Pinker MD}
\author[2\authfn{1}]{P.~Baltzer MD}
\author[2\authfn{1}]{P.~Andrzejewski MD}
\author[1\authfn{1}]{J.~Massich PhD}

\contrib[\authfn{1}]{Equally contributing authors.}

% Include full affiliation details for all authors
\affil[1]{Department, Institution, City, State or Province, Postal Code, Country}
\affil[2]{Department, Institution, City, State or Province, Postal Code, Country}

\corraddress{Author One PhD, Department, Institution, City, State or Province, Postal Code, Country}
\corremail{correspondingauthor@email.com}

\presentadd[\authfn{2}]{Department, Institution, City, State or Province, Postal Code, Country}

\fundinginfo{Funder One, Funder One Department, Grant/Award Number: 123456, 123457 and 123458; Funder Two, Funder Two Department, Grant/Award Number: 123459}

% Include the name of the author that should appear in the running header
\runningauthor{Author One et al.}

\input{file_system/acronym_definition.tex}

\begin{document}

\maketitle


\begin{abstract}
    Study the advantages of using multi-modal imagery in computer-aided
    detection and diagnosis systems for prostate cancer with special attention to
    pre-processing and data-balancing stages of a CaD.
%
    A multi modal MR volumes including T2W-MRI, DCE-MRI, DW-MRI and MRSI with
    accompanying ground-truth were acquired from 17 patients. The proposed CaD
    system to conduct this study consists of pre-processing, segmentation, feature
    detection, data balancing, feature extraction and classification;
    and all results are cross-validated using leave one patient out strategy.
%
    The usage of multi-modal information to the CaD system increases the area
    under the curve from $0.666\pm0.15$ up to $0.836\pm0.083$, showing a better
    and more stable results.
%
    The study using the proposed framework shows that T2W-MRI is necessary to
    obtain satisfactory results however better results are obtained when combined
    with other modalities.
    % More-over the study also shows that the reliability of any modality is
    % strongly linked to the pre-processing and balancing of the data.

% Please include a maximum of seven keywords
\keywords{Computer-aided detection, prostate cancer, multi-parametric MRI, feature crafting}
\end{abstract}




\section{Introduction}

\Ac{cap} is the second most frequently diagnosed cancer in men, accounting for
899,000 cases and leading to 258,100 deaths per year~\citep{ferlay2010estimates}.
%
Early detection, diagnosis and accurate risk assessment play a major role in
patient treatment. In this regard, \citeauthor{Hambrock2013}\,\cite{Hambrock2013}
explore the idea of incorporating \ac{cad} systems to assist radiologists in
their clinical practice; showing that applying \ac{cad} to \ac{cap} detection
and diagnosis improve the clinician's performance specially for inexperienced
radiologists which when assisted by \ac{cad} system are able to obtain results
equivalent to those from a senior radiologist.

Despite recent advances, in \ac{cap}-\ac{cad} remains a young technology mainly
because it is based on \ac{mri}~\cite{Hegde2013}.
%
The two main challenges in \ac{cap}-\ac{cad} systems, as pointed by the
\ac{pirads} Steering Committee are: (i) improving the detection of clinically
significant \ac{cap} and (ii) increasing confidence in benign or dormant cases
to avoid unnecessary invasive medical exams~\citep{weinreb2016pi}.

\citeauthor{lemaitre2015computer}\,\cite{lemaitre2015computer} provide an
overview and categorization of the developed \acs{cad} systems for \ac{cap}
detection and diagnosis based on \ac{mri} modalities.
The main highlights can be
found in \acs{tab}\,\ref{tab:survey-summary};
whereas
\acs{fig}\,\ref{fig:wkfcad} correspond to the common \ac{cad} framework
identified in~\cite{lemaitre2015computer}, which is detailed in \acs{sec}\,\ref{sec:framework}.

The main goal of this work is to evaluate the impact of using \ac{mpmri} in
\ac{cap}-\ac{cad} systems and also bring some light to the following key aspects:
(i) normalization of \ac{dce}-\ac{mri} modality,
(ii) the effect of balancing the data,
and (iii) thorough study in feature selection/extraction.

The rest of this document is structured as follows:
\acs{sec}\,\ref{sec:materials_and_methods} presents the materials and methods used in
this study including details of the data, the \ac{cad} framework and its
implementation; \acs{sec}\,\ref{sec:experiments} presents the experiments; and
\ac{sec}\,\ref{sec:conclusion} presents the conclusions and further work.



% survey's summary
\begin{figure}
\begin{itemize} 
\item summary: 
  \begin{itemize} 
    \item 56 studies
    \item num of MRI modalities: 1 - 33\%, 2 - 31\%, 3 - 33\%, 4 - 1\%
    \item Scanner type: 3T - 60\%, 1.5T - 40\%
    \item studied zones: PZ - 45\%, PZ+CG 55\%
    \item cad type: cade-cadx 70\%, cadx 30\%
  \end{itemize}
\item positive:
  \begin{itemize} 
    \item 3 modalities better than 2
    \item Texture and edge features are predominant
    \item Features selection/extraction tends to improve performance
    \item Pre-eminence of SVM and ensemble classifier (i.e., AdaBoost, RF, etc.)
  \end{itemize}
\item negative:
  \begin{itemize} 
    \item no publicly available mp-mri dataset
    \item only 1 study used 4 mri modalities
    \item limited work on data normalization
    \item a lot of features are extracted in 2d
    \item limited work regarding selection/extraction
    \item no work regarding data balancing
    \item no source code available of any cad
  \end{itemize}
\end{itemize}
\caption{Survey's summary}
\label{tab:survey-summary}
\end{figure}




\section{Materials and Methods} \label{sec:materials_and_methods}

\subsection{Data}\label{sec:data}

All experiments are conducted on a subset of public \ac{mpmri} prostate dataset
available in~\citep{lemaitre2016dce}. The subset used in this work corresponds
to the 17 \ac{cap} cases proven positive by biopsy. The images were acquired
with a \SI{3}{\tesla} whole body \ac{mri} scanner (Siemens Magnetom Trio TIM,
Erlangen, Germany) using sequences to obtain \ac{t2w}-\ac{mri},
\ac{dce}-\ac{mri} and \ac{dw}-\ac{mri}. This last one is used to compute the
\ac{adc} maps. All cases have complementary \ac{mrsi} and \ac{gt} data.
%
For each case the \ac{gt} is composed by the following volume-delinations
provided by experienced radiologits: the prostate on \ac{t2w}, \ac{adc} and for
a single \ac{dce}-\ac{mri} time series; the prostate's \ac{cg} on \ac{t2w}; plus
all the lesion delineations, also on \ac{t2w}. A total of 5 volumes per case.


\subsection{CaP framework} \label{sec:framework}

\Acl{fig}~\ref{fig:wkfcad} illustrates the common \ac{cad}-\ac{cap} workflow
resulting from the State-of-the-Art~\cite{lemaitre2015computer}, which 
consists of three distinctive processes: (i) image regularization, (ii)
\ac{cade} and (iii) \ac{cadx}. Image regularisation focuses on formatting the data
while \ac{cade} and \ac{cadx} allow to detect possible lesions and distinguish
malignant from non-malignant tumors, respectively.

\begin{figure}
  \centering
\input{./images/cad/cad_1.tex}
\caption{framework}
\label{fig:wkfcad}
\end{figure}

Three types of pre-processing are applied to \ac{mri} images: (i) noise filtering,
(ii) bias correction, and (iii) standardization/normalization; whereas
\ac{mrsi} data is processed to correct the phase, baseline and frequency
as in \citeauthor{Chen2002}\,\cite{Chen2002}.
%
This study does not cover automatic prostate segmentation, therefore manual
delineations from the \ac{gt} drive the registration where needed.
The registration of the data is done as follows: (i) \ac{dce}-\ac{mri} volumes
are rigidly registered within themselves, to remove any patient motion during
the acquisition; (ii) non-rigid registration is performed between the
\ac{t2w}-\ac{mri} and the rest of modalities.
%
In terms of \ac{cade}, for this study all the data within the prostate is
considerate a potential lesion.
%
For feature detection, the complete set of features reported in the 56 studies
covered by \cite{lemaitre2015computer} has been implemented.
\Ac{tab}\,\ref{tab:feat} summarizes such features indicating its reported presence.
%
Data balancing stage is added to ensure proper learning in a scenario where the
number of negative samples is greater than the positive, as this one.
%
\ac{rf} is used in feature selection and classification steps. This choice is
motivated by the fact that
(i) comparison of different classifiers is out of the
scope of this work;
(ii) \ac{rf} is a suited classifier for the task in
hand~\cite{lemaitre2015computer};
and (iii) the capability to interpret the feature selection when performed using \ac{rf}.

% \subparagraph{Feature selection}
% \emph{Several feature extraction strategies are studied.}
% \subparagraph{classifier choice}
% \emph{
%   There's no interest in this work in comparing one classifier against others,
%   we just went for \ac{rf} since based of the state-of-the-art it works good and
%   bring us some insights regarding the feature selection etc..
%   \Ac{rf} is used as classifier, so that we can take advantage of feature
%   selection combination etc... we also use adaboost and gradient boosting to
%   create meta classifiers.
% %
%   Among those, \ac{rf} showed its reliability to lead to high classification performance.
%   That is why, \ac{rf} has been chosen as our base classifier --- allowing for feature selection as well --- to perform classification of individual modality as well as the combination of modalities.
% }

\subparagraph{Other considerations:}
(i) All the experiments are carried out using \ac{lopo}.
(ii) The implementation of the registration (C++), normalization (Python), and
classification pipeline (Python) are publicly available on
GitHub\footnote{\url{https://github.com/I2Cvb/lemaitre-2016-nov/tree/master}}~\citep{lemaitre2016github}.

\begin{table*}
  \centering
  \caption{Overview of the feature detection methods used in \ac{cad} systems.}\label{tab:feat}
  \footnotesize
  \begin{threeparttable}
    \renewcommand{\arraystretch}{.7}
    \begin{tabular}{p{.5\linewidth} p{.4\linewidth}}
      \hline \\ [-1.5ex]
      \textbf{Feature detection methods} & \textbf{Number of works using this feature} \\ \\ [-1.5ex]
      \hline \\ [-1.5ex]
      \textbf{\ac{mri} image:} & \\ \\ [-1.5ex]
      \quad \textit{Voxel-wise detection} &  \\ \\ [-1.5ex]
      \quad \quad Intensity-based & 16 \\ %0.285714285714 \\ %16/56
      \quad \quad Edge-based & \\
      \quad \quad \quad Prewitt operator & 4 \\
      \quad \quad \quad Sobel operator & 9 \\
      \quad \quad \quad Kirsch operator & 9 \\
      \quad \quad \quad Gabor filtering & 3 \\
      \quad \quad Texture-based & \\
      \quad \quad \quad Haralick features & 11 \\
      \quad \quad \quad Fractal analysis & 2 \\
      \quad \quad \quad \Ac{dct} & 1 \\
      \quad \quad \quad Wavelet-based features & 1 \\
      \quad \quad \quad Gaussian filter bank & 1 \\
      \quad \quad Position-based &  4 \\ \\ [-1.5ex]
      \quad \textit{Region-wise detection} &  \\ \\ [-1.5ex]
      \quad \quad Statistical-based & \\
      \quad \quad \quad Percentiles & 10 \\
      \quad \quad \quad Statistical-moments & 16 \\
      \quad \quad Histogram-based & \\
      \quad \quad \quad \acs{pdf} & 1 \\
      \quad \quad \quad \acs{hog} & 1 \\
      \quad \quad \quad Shape context & 1 \\
      \quad \quad \quad \acs{lbp} & 1 \\
      \quad \quad Anatomical-based & 3 \\
      \textbf{\ac{dce} signal:} & \\ \\ [-1.5ex]
      \quad Whole spectra approach & 2 \\
      \quad Semi-quantitative approach & 5 \\
      \quad Quantitative approach &  \\
      \quad \quad Toft model & 7 \\
      \quad \quad Brix model & 6 \\
      \quad \quad Weibull function & 2 \\
      \quad \quad PUM & 2 \\
      \\ [-1.5ex]
      \textbf{\ac{mrsi} signal:} & \\ \\ [-1.5ex]
      \quad Whole spectra approach & 10 \\
      \quad Quantification approach & 1 \\
      \quad Wavelet-based approach & 1 \\
      \hline
    \end{tabular}
    % \begin{tablenotes}
    % \end{tablenotes}
  \end{threeparttable}
\end{table*}


% Our \ac{mpmri} \ac{cad} system consists of seven different steps: pre-processing, segmentation, registration, feature detection, balancing, feature selection/extraction, and finally classification.

% \paragraph{Pre-processing}
% Three types of pre-processing are used for \ac{mri} images: (i) noise filtering, (ii) bias correction, and (iii) standardization/normalization.

% Additionally, the \ac{mrsi} modality requires a specific pre-processing based on signal processing rather than image processing.
% Therefore, the \ac{mrsi} modality has been pre-processed to correct the phase, baseline, and frequency.
% Additionally, each \ac{mrsi} spectrum is normalized using the L$_2$ norm, which has been shown to be the most efficient normalization method in \ac{mrsi} as discussed in \acs{sec}\,\ref{subsec:chp3img-reg:prepro}.
% \paragraph{segmentation}
% For this study, no segmentation method has been developed and the manual segmentation given by our radiologist has been used.
% \paragraph{registration}
% \paragraph{balancing}
% \paragraph{feature detection}

% To approach the task of automatic detection of \ac{cap} using machine learning, one has to extract a variety of feature specific to the \ac{mri} modality as presented in \acs*{sec}\,\ref{subsec:chp3:img-clas:CADX-fea-dec}.

% \subparagraph{\ac{t2w}-\ac{mri} and \ac{adc} map features}
% \Acl{tab}~\ref{tab:featureadct2w} summarizes the different features extracted with their corresponding parameters.
% Note that all these features are extracted at each voxel of the volume.
% \begin{table}
%   \caption{Features extracted in \acs*{t2w}-\acs*{mri} and \acs*{adc} volumes.}
%   \centering
%   \scriptsize
%   \begin{tabularx}{\textwidth}{lXc}
%     \toprule
%     \textbf{Features} & \textbf{Parameters} & \textbf{\# dimensions} \\
%     \midrule
%     Intensity &  & 1 \\
%     \acs*{dct} decomposition & window: \SI[product-units=repeat]{9x9x3}{\px} & 243 \\
%     Kirsch filter &  & 2 \\
%     Laplacian filter &  & 1 \\
%     Prewitt filter &  & 3 \\
%     Scharr filter &  & 3 \\
%     Sobel filter &  & 3 \\
%     Gabor filters & 4 frequencies $f \in [0.05, 0.25]$; 4 azimuth angles $\alpha \in [0, \pi]$; 8 elevation angles $\alpha \in [0, 2\pi]$ & 256 \\
%     Phase congruency filter & 5 orientations; 6 scales & 3 \\
%     Haralick filter & window: \SI[product-units=repeat]{9x9x3}{\px}; \# grey levels: 8; distance: \SI{1}{\px}; 13 directions & 169 \\
%     \acs*{lbp} filter & 2 radii $r=\{1, 2\}$; 2 neighborhood sizes $N = \{8, 16\}$ & 6 \\
%     \bottomrule
%   \end{tabularx}
%   \label{tab:featureadct2w}
% \end{table}

% To approach the task of automatic detection of \ac{cap} using machine learning, one has to extract a variety of feature specific to the \ac{mri} modality as presented in \acs*{sec}\,\ref{subsec:chp3:img-clas:CADX-fea-dec}.

% \paragraph{Feature extraction}
% Feature selection and extraction are used in the experiment: (i) signal-based data --- i.e., \ac{mrsi} and \ac{dce}-\ac{mri} --- are decomposed using feature extraction methods while (ii) image-based features are selected through different feature selection methods.
% These methods have been presented in \acs{sec}\,\ref{subsec:chp3:img-clas:CADX-fea-ext}.

% Among those, \ac{pca}, sparse-\ac{pca}, and \ac{ica} are used to decompose signal-based data.
% \paragraph{classification}
% Variety of classifiers have been explained in \acs{sec}\,\ref{subsec:chp3:img-clas:CADX-clas}. 

% Among those, \ac{rf} showed its reliability to lead to high classification performance.
% That is why, \ac{rf} has been chosen as our base classifier --- allowing for feature selection as well --- to perform classification of individual modality as well as the combination of modalities.

% % \begin{figure}
% %   \centering
% %   \includegraphics[width=0.5\linewidth]{content/figures/stacking_gb.png}
% %   \caption[The principle of stacking.]{The principle of stacking. First,
% %     training samples (red) are used to train each individual \ac{rf}.
% %     Subsequently, a validation set (green) is provided to each \ac{rf} which
% %     outputs a set of probabilities used for the classification of the
% %     meta-classifier. Finally, a test set is used to asses the classification
% %     performance to whole stack.}
% %   \label{fig:stacking}
% % \end{figure}

% Additionally, we use stacking to create ensemble of base learners using a meta-classifier~\cite{wolpert1992stacked}.
% \Acl{fig}~\ref{fig:stacking} illustrate the principle of stacking.
% Stacking consists in a two-stage learning:
% (i) First, a set of training samples is used to train each individual base learner and
% (ii) subsequently, a set of validation samples is provided to each \ac{rf} which individually output a corresponding set of probability used to train a meta-classifier.
% Finally, the stack of classifiers is assessed by providing a test set which is going through the base learners and the meta-classifier.

% In the later experiments, \ac{adb} and \ac{gb} are chosen as meta-classifiers to aggregate the base learners in the stacking strategies.
% The difference between \ac{adb} and \ac{gb} lie in the fact the this
% minimization procedure.

\section{Experiment and results}
\label{sec:experiments}

\Acl{sec}\,\ref{sec:experiments:mpmri-comparison} analyzes the impact of \ac{mpmri}, which
is summarized in \Ac{fig}\,\ref{fig:modality-combination} where the results from
individual modalities are compared to different strategies for combining the
information brought by each modality.
Whereas, \ac{sec}\,\ref{sec:experiments:light} discusses some particular aspects
of the workflow in \Ac{sec}\,\ref{sec:framework} in order to bring each
modality to its best light in order to compose \Ac{fig}\,\ref{fig:modality-combination:separeate_modalities}.

\subsection{\ac{cap}-\ac{cad} bits}
\label{sec:experiments:mpmri-comparison}
\paragraph{The value of supplying \ac{dce} data directly}
\paragraph{For the case of \ac{dce} data, a good normalization is all you need.}

The features derived from the \ac{dce} modality can be categorized in: 
(i) quantitative methods which describe the signal based on parameter values
better fitting a given pharmacokinetic model such as Toft or Brix;
(ii) semi-quantitative methods that use some descriptor such as wash in and wash
out to embed the signal;
or (iii) the entire signal.

Despite literature favoring the first two group of
features~\cite{lemaitre2015computer} \emph{because they integrate over the
  singal independently}, we argue that a proper normalization of
the entire signal that brings \emph{all the patients, all the under the same
  coordinate system} leads to better results. 
%
\Acl{fig}\,\ref{fig:DCE-norm} compares the quantitative and semi-quantitative
features against the entire \ac{dce} signal with (and without) normalization,
showing that properly normalized data brings the most discriminating information
from \ac{dce} modality.

% \subparagraph{\ac{dce}-\ac{mri} features}
% In brief, the entire enhanced signal, semi-quantitative, and quantitative methods are computed.
% The reader can refer to \ac{sec}\,\ref{subsubsec:chp5:DCE-norm:stateart} for a detailed presentation of the different methods used.

\begin{figure}
  \hspace*{\fill}
  \subfigure[Performance of the quantitative methods on \acs*{dce}.] {
    \includegraphics[width=.45\textwidth]
    {5_normalization/figures/DCE-normalization/normalized_methods_0.pdf}
  }
  \hfill
  \subfigure[Performance of enhanced \acs*{dce} signal.]{
  \includegraphics[width=.45\textwidth]
  {5_normalization/figures/DCE-normalization/full_signal_0.pdf}
  }
  \hspace*{\fill}
  \caption[DCE] {DCE}
  \label{fig:DCE-norm}
\end{figure}

% \subparagraph{\ac{mrsi} features}
% Due to unavailability of some unsuppressed water acquisition, absolute quantification as presented by \citeauthor{trigui2017automatic} could not be computed~\cite{trigui2017automatic}.
% Therefore, likewise in~\cite{Parfait2012}, three different techniques are used to extract discriminative features: (i) relative quantification based on metabolite quantification, (ii) relative quantification based on bounds integration, and (iii) spectra extraction.
% \subparagraph{Anatomical features}
% Beside the aforementioned features specific at each modality, anatomical features as proposed by \citeauthor{Chen2002} and \citeauthor{Litjens2014} are computed~\cite{Chen2002,Litjens2014}.
% Therefore, 4 different metrics are computed based on the relative distance to the prostate boundary as well as the prostate center, and the relative position in the Euclidean and cylindrical coordinate systems.

\section{The advantage of mpmri}%\ac{mpmri}}
\label{sec:experiments:mpmri-comparison}

As aforementioned, \ac{fig}\,\ref{fig:modality-combination} compares the best
results of each individual image modality (\ac{t2w}-\ac{mri}, \ac{dce}-\ac{mri},
\ac{adc} map and \ac{mrsi}) in 
\ac{fig}\,\ref{fig:modality-combination:separeate_modalities}) against three
different manners of combining them in \ac{fig}\,\ref{fig:modality-combination:combined}:
(i) in blue, a feature vector composed with the selected features of each image
modality (those selected to produce
\Ac{fig}\,\ref{fig:modality-combination:separeate_modalities}) is fitted to a
\ac{rf} classifier.
(ii) In green, the classifiers producing
\Ac{fig}\,\ref{fig:modality-combination:separeate_modalities} are used to train
a stacking classifier with a \ac{gb} as meta-classifier.
And (iii) in red, all the features from all modalities (see
\Ac{tab}\,\ref{tab:feat}) concatenated as a single vector are directly fitted to
a \ac{rf} classifier.

The best results are achieved using the last configuration with an \ac{auc} of
$0.836 \pm 0.083$.
\Acl{fig}\,\ref{fig:resultcad} illustrates qualitative results of this
configuration by overlapping the probability map of having a \ac{cap} with the
original \ac{t2w}-\ac{mri} slice, for 6 diverse patients.
\emph{Something more comparing to single modality plot.}

\begin{figure}
  \hspace*{\fill}
  \subfigure[] {
    \includegraphics[width=.45\textwidth]
    {content/figures/exp-1/all.pdf}
    \label{fig:modality-combination:separeate_modalities}
  }
  \hfill
  \subfigure[Analysis of feature combination approaches after fine tuning.]{
    \includegraphics[width=.45\textwidth]
    {content/figures/exp-5/combine_all.pdf}
    \label{fig:modality-combination:combined}
  }
  \hspace*{\fill}
  \caption[modality combination] {Modality combination}
  \label{fig:modality-combination}
\end{figure}

Due to the limited body of work using \ac{mrsi} data in \ac{cap}-\ac{cad}
systems~\cite{lemaitre2015computer}, we also compare the stacking strategy with
and without \ac{mrsi} data in order to see its influence. We observe that with
no \ac{mrsi} data the final performance drops from 
$0.786 \pm 0.098$ to $0.756 \pm 0.092$.
This aligns with the fact that $29.2\%$ of the features selected when
concatenating all the features from all modalities are in fact from \ac{mrsi}
data (see \Ac{tab}\,\ref{tab:selfeatocc}).


\section{Conclusion}
\label{sec:conclusion}

% this review has presented an overview and classification of the research related
% to \ac{cad} development for \ac{cap} using multi-parametric \ac{mri} data. we
% aimed at providing background information regarding multi-parametric \ac{mri}
% imaging techniques and a description of the work-flow in the different \ac{cad}
% stages. the methods used in the literature for each of these stages have been
% reviewed along with the available results of the \ac{cad} systems. moreover,
% insight discussions and possible future research directions have also been
% given. finally, a multi-parametric multi-vendor dataset has been made available
% to the research community in order to provide a standardised platform for
% \ac{cad} development and evaluation for \ac{cap} using multi-parametric
% \ac{mri}.

% In this regard, \ac{mpmri} is frequently
% used to build robust \ac{cad} systems to detect, localize, and grade \ac{cap}.
% In general, \ac{cad} systems are based on \ac{mpmri} which potentially combines
% several of the following modalities~\citep{lemaitre2015computer}:
% \ac{t2w}-\ac{mri}, \ac{dce}-\ac{mri}, \ac{adc} maps, and \ac{mrsi}.

\begin{table}
  \caption{Selected feature and number of occurrence for \acs*{t2w}-\acs*{mri}, \acs*{adc} map, and one all the features are concatenated.}
  \centering
  \scriptsize
  \begin{tabular}{llllll}
    \toprule
    \multicolumn{1}{c}{\textbf{\acs*{t2w}-\acs*{mri}}} & \multicolumn{1}{c}{\textbf{\acs*{adc}}} & \multicolumn{1}{c}{\textbf{\acs*{t2w}-\acs*{mri}}} & \multicolumn{1}{c}{\textbf{\acs*{adc}}} & \multicolumn{1}{c}{\textbf{\acs*{dce}-\acs*{mri}}} & \multicolumn{1}{c}{\textbf{\acs*{mrsi}}} \\
    \cmidrule(lr){1-1} \cmidrule(lr){2-2} \cmidrule(lr){3-6}
    8 edges & 1 \acs*{dct} & 113 Gabor filters & 53 Gabor filters & 14 samples  & 78 samples \\
    155 Gabor filters & 32 Gabor filters & 1 phase congruency & 2 phase congruency & & \\ 
    2 Haralick features & 1 phase congruency & 4 edges & & & \\
    1 intensity & & 1 intensity & & & \\
    4 \acs*{lbp} & & & & & \\
    2 phase congruency & & & & & \\
    \cmidrule(lr){1-1} \cmidrule(lr){2-2} \cmidrule(lr){3-6}
    \multicolumn{1}{c}{\textbf{172 features}} & \multicolumn{1}{c}{\textbf{34 features}} & \multicolumn{4}{c}{\textbf{267 features}} \\
    \bottomrule
  \end{tabular}
  \label{tab:selfeatocc}
\end{table}


\begin{landscape}
\begin{figure}
  \hspace*{\fill}
  \subfigure[\acs*{auc} = 0.922]{\label{fig:pat634}\includegraphics[width=.45\textwidth]{content/figures/examples/patient_634.pdf}}
  \hfill
  \subfigure[\acs*{auc} = 0.942]{\label{fig:pat778}\includegraphics[width=.45\textwidth]{content/figures/examples/patient_778.pdf}}
  \hfill
  \subfigure[\acs*{auc} = 0.914]{\label{fig:pat1036}\includegraphics[width=.45\textwidth]{content/figures/examples/patient_1036.pdf}}
  \hspace*{\fill}\\
  \hspace*{\fill}
  \subfigure[\acs*{auc} = 0.692]{\label{fig:pat634}\includegraphics[width=.45\textwidth]{content/figures/examples/patient_410.pdf}}
  \hfill
  \subfigure[\acs*{auc} = 0.879]{\label{fig:pat778}\includegraphics[width=.45\textwidth]{content/figures/examples/patient_784.pdf}}
  \hfill
  \subfigure[\acs*{auc} = 0.735]{\label{fig:pat1036}\includegraphics[width=.45\textwidth]{content/figures/examples/patient_1041.pdf}}
  \hspace*{\fill}
  \caption[Illustration the resulting detection of our \acs*{mpmri} \acs*{cad} for \acs*{cap} detection.]{Illustration the resulting detection of our \acs*{mpmri} \acs*{cad} for \acs*{cap} detection. The blue contours corresponds to the \ac{cap} while the \texttt{jet} overlay represents the probability.}
  \label{fig:resultcad}
\end{figure}
\end{landscape}

% \input{./content/chapter6.tex}


Therefore, including \ac{mrsi} into the classification pipeline increases the \ac{auc} from $0.756 \pm 0.092$ to $0.786 \pm 0.098$ for a gain of $0.030$.

%%%%%%%%%%%%%%%%%%%% End comment



% \begin{table[bt]
% \caption{This is a table. Tables should be self-contained and complement, but not duplicate, information contained in the text. They should be not be provided as images. Legends should be concise but comprehensive – the table, legend and footnotes must be understandable without reference to the text. All abbreviations must be defined in footnotes.}
% \begin{threeparttable}
% \begin{tabular}{lccrr}
% \headrow
% \thead{Variables} & \thead{JKL ($\boldsymbol{n=30}$)} & \thead{Control ($\boldsymbol{n=40}$)} & \thead{MN} & \thead{$\boldsymbol t$ (68)}\\
% Age at testing & 38 & 58 & 504.48 & 58 ms\\
% Age at testing & 38 & 58 & 504.48 & 58 ms\\
% Age at testing & 38 & 58 & 504.48 & 58 ms\\
% Age at testing & 38 & 58 & 504.48 & 58 ms\\
% \hiderowcolors
% stop alternating row colors from here onwards\\
% Age at testing & 38 & 58 & 504.48 & 58 ms\\
% Age at testing & 38 & 58 & 504.48 & 58 ms\\
% \hline  % Please only put a hline at the end of the table
% \end{tabular}

% \begin{tablenotes}
% \item JKL, just keep laughing; MN, merry noise.
% \end{tablenotes}
% \end{threeparttable}
% \end{table}

% \section*{acknowledgements}
% Acknowledgements should include contributions from anyone who does not meet the criteria for authorship (for example, to recognize contributions from people who provided technical help, collation of data, writing assistance, acquisition of funding, or a department chairperson who provided general support), as well as any funding or other support information.

% \section*{conflict of interest}
% The authors report no potential conflicts of interest. 

% \printendnotes

% Submissions are not required to reflect the precise reference formatting of the journal (use of italics, bold etc.), however it is important that all key elements of each reference are included.
% \bibliography{sample}
\bibliography{literature_review_2}

\begin{biography}[example-image-1x1]{A.~One}
Please check with the journal's author guidelines whether author biographies are required. They are usually only included for review-type articles, and typically require photos and brief biographies (up to 75 words) for each author.
\bigskip
\bigskip
\end{biography}

\graphicalabstract{example-image-1x1}{Please check the journal's author guildines for whether a graphical abstract, key points, new findings, or other items are required for display in the Table of Contents.}

\end{document}
