%%%%%%%%%%%%%%%%%%%%%%%%%%%%%%%%%%%%%%%%%%%%%%%%%%%%%%%
% A template for Wiley article submissions.
% Developed by Overleaf. 
%
% Please note that whilst this template provides a 
% preview of the typeset manuscript for submission, it 
% will not necessarily be the final publication layout.
%
% Usage notes:
% The "blind" option will make anonymous all author, affiliation, correspondence and funding information.
% Use "num-refs" option for numerical citation and references style.
% Use "alpha-refs" option for author-year citation and references style.

\documentclass[num-refs]{wiley-article}
% \documentclass[blind,alpha-refs]{wiley-article}

% Add additional packages here if required
\input{file_system/package.tex}

% Update article type if known
\papertype{Original Article}
% Include section in journal if known, otherwise delete
\paperfield{Journal Section}

\title{Computer-aided detection and diagnosis using Multi-modal MRI for prostate cancer detection}

% List abbreviations here, if any. Please note that it is preferred that abbreviations be defined at the first instance they appear in the text, rather than creating an abbreviations list.
% \abbrevs{ABC, a black cat; DEF, doesn't ever fret; GHI, goes home immediately.}

% Include full author names and degrees, when required by the journal.
% Use the \authfn to add symbols for additional footnotes and present addresses, if any. Usually start with 1 for notes about author contributions; then continuing with 2 etc if any author has a different present address.
% \author[1\authfn{1}]{Author One PhD}
% \author[2\authfn{1}]{Author A.~Two MD}
% \author[2\authfn{2}]{Author Three PhD}
% \author[2]{Author B.~Four}
\author[1\authfn{1}]{G.~Lemaitre PhD}
\author[2\authfn{1}]{F.~Meriaudeau PhD}
\author[2\authfn{1}]{A.~Meyer-Bease PhD}
\author[2\authfn{1}]{R.~Marti PhD}
\author[2\authfn{1}]{K.~Pinker MD}
\author[2\authfn{1}]{P.~Baltzer MD}
\author[2\authfn{1}]{P.~Andrzejewski MD}
\author[1\authfn{1}]{J.~Massich PhD}

\contrib[\authfn{1}]{Equally contributing authors.}

% Include full affiliation details for all authors
\affil[1]{Department, Institution, City, State or Province, Postal Code, Country}
\affil[2]{Department, Institution, City, State or Province, Postal Code, Country}

\corraddress{Author One PhD, Department, Institution, City, State or Province, Postal Code, Country}
\corremail{correspondingauthor@email.com}

\presentadd[\authfn{2}]{Department, Institution, City, State or Province, Postal Code, Country}

\fundinginfo{Funder One, Funder One Department, Grant/Award Number: 123456, 123457 and 123458; Funder Two, Funder Two Department, Grant/Award Number: 123459}

% Include the name of the author that should appear in the running header
\runningauthor{Author One et al.}

\input{file_system/acronym_definition.tex}

\begin{document}

\maketitle


\begin{abstract}
    Study the advantages of using multi-modal imagery in computer-aided
    detection and diagnosis systems for prostate cancer with special attention to
    pre-processing and data-balancing stages of a CaD.
%
    A multi modal MR volumes including T2W-MRI, DCE-MRI, DW-MRI and MRSI with
    accompanying ground-truth were acquired from 17 patients. The proposed CaD
    system to conduct this study consists of pre-processing, segmentation, feature
    detection, data balancing, feature extraction and classification;
    and all results are cross-validated using leave one patient out strategy.
%
    The usage of multi-modal information to the CaD system increases the area
    under the curve from $0.666\pm0.15$ up to $0.836\pm0.083$, showing a better
    and more stable results.
%
    The study using the proposed framework shows that T2W-MRI is necessary to
    obtain satisfactory results however better results are obtained when combined
    with other modalities.
    % More-over the study also shows that the reliability of any modality is
    % strongly linked to the pre-processing and balancing of the data.

% Please include a maximum of seven keywords
\keywords{Computer-aided detection, prostate cancer, multi-parametric MRI, feature crafting}
\end{abstract}


\section{Introduction}

\Ac{cap} is the second most frequently diagnosed cancer in men, accounting for
899,000 cases and leading to 258,100 deaths per
year~\citep{ferlay2010estimates}. As highlighted by the PI-RADS Steering
Committee, the two main challenges to be addressed are~\citep{weinreb2016pi}:
(i) improving the detection of clinically significant \ac{cap}
and (ii) increasing confidence in benign or dormant cases and therefore avoiding
unnecessary invasive medical exams. In this regard, \ac{mpmri} is frequently
used to build robust \ac{cad} systems to detect, localize, and grade \ac{cap}.
In general, \ac{cad} systems are based on \ac{mpmri} which potentially combines
several of the following modalities~\citep{lemaitre2015computer}:
\ac{t2w}-\ac{mri}, \ac{dce}-\ac{mri}, \ac{adc} maps, and \ac{mrsi}.

\emph{Semi-quantitative} approaches are mathematical rather than pharmacokinetic because no pharmacokinetic assumptions regarding the relationship between the \ac{mri} signal and the contrast agent are made~\citep{huisman2001accurate,gliozzi2011phenomenological}.
These methods are advantageous because they do not require any knowledge of the \ac{mri} sequence or any conversion from signal intensity to concentration.
However, they present some limitations: the heuristic approach
proposed by \citeauthor{huisman2001accurate} requires an initial
estimate of the standard deviation of the signal noise and some manual tuning.

Nevertheless, all of the presented methods suffer from the following two major drawbacks:
(i) inter-patient variability and (ii) loss of information.
The inter-patient variability is mainly due to the acquisition process
and consequently leads to generalization issues in applying a machine learning algorithm.
All previous methods extract few discriminative parameters to describe the \ac{dce}-\ac{mri} signal which might lead to a loss of information.
%(i) the inter-patient variability of the data lead to a variation of the parameters estimated and subsequently to poor classification performance while designing \ac{cad} systems, and
%(ii) only few parameters are used to characterize the dynamic signal implying that some information are discarded.

In this work, we propose a fully automatic normalization method for \ac{dce}-\ac{mri} that reduces the inter-patient variability of the data.
The benefit and simplicity of our approach will be demonstrated by
classifying the whole normalized \ac{dce}-\ac{mri} signal and
comparing it with state-of-the-art quantitative and semi-quantitative methods.
Additionally, we will show that using this normalization approach in conjunction with the quantitative methods improves the classification performance of most of the models.
We also propose a new clustering-based method to discern enhanced
signals from the arteries that can, later be used to estimate an
\ac{aif} and provide an alternative approach to estimate the parameters of the semi-quantitative model proposed by~\cite{huisman2001accurate}.


\input{./content/chapter6.tex}


% \begin{table}[bt]
% \caption{This is a table. Tables should be self-contained and complement, but not duplicate, information contained in the text. They should be not be provided as images. Legends should be concise but comprehensive – the table, legend and footnotes must be understandable without reference to the text. All abbreviations must be defined in footnotes.}
% \begin{threeparttable}
% \begin{tabular}{lccrr}
% \headrow
% \thead{Variables} & \thead{JKL ($\boldsymbol{n=30}$)} & \thead{Control ($\boldsymbol{n=40}$)} & \thead{MN} & \thead{$\boldsymbol t$ (68)}\\
% Age at testing & 38 & 58 & 504.48 & 58 ms\\
% Age at testing & 38 & 58 & 504.48 & 58 ms\\
% Age at testing & 38 & 58 & 504.48 & 58 ms\\
% Age at testing & 38 & 58 & 504.48 & 58 ms\\
% \hiderowcolors
% stop alternating row colors from here onwards\\
% Age at testing & 38 & 58 & 504.48 & 58 ms\\
% Age at testing & 38 & 58 & 504.48 & 58 ms\\
% \hline  % Please only put a hline at the end of the table
% \end{tabular}

% \begin{tablenotes}
% \item JKL, just keep laughing; MN, merry noise.
% \end{tablenotes}
% \end{threeparttable}
% \end{table}

% \section*{acknowledgements}
% Acknowledgements should include contributions from anyone who does not meet the criteria for authorship (for example, to recognize contributions from people who provided technical help, collation of data, writing assistance, acquisition of funding, or a department chairperson who provided general support), as well as any funding or other support information.

% \section*{conflict of interest}
% The authors report no potential conflicts of interest. 

% \printendnotes

% Submissions are not required to reflect the precise reference formatting of the journal (use of italics, bold etc.), however it is important that all key elements of each reference are included.
% \bibliography{sample}
\bibliography{literature_review_2}

\begin{biography}[example-image-1x1]{A.~One}
Please check with the journal's author guidelines whether author biographies are required. They are usually only included for review-type articles, and typically require photos and brief biographies (up to 75 words) for each author.
\bigskip
\bigskip
\end{biography}

\graphicalabstract{example-image-1x1}{Please check the journal's author guildines for whether a graphical abstract, key points, new findings, or other items are required for display in the Table of Contents.}

\end{document}
